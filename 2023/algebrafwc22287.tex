\begin{enumerate}
\item \textbf{Assertion (A):}The polynomial p(x)=$x^{2}+3x+3$ has two real zeroes.
	\\\textbf{Reason (R) :} A quadratic polynomial can have at most two zeroes.
\begin{enumerate}
\item Both Assertion (A) and Reason (R) are true and Reason (R) is the correct explanation of Assertion (A). 
\item Both Assertion (A) and Reason (R) are true and Reason (R) is not the correct explanation of Assertion (A).
\item Assertion (A) is true but Reason (R) is false.
\item Assertion (A) is false but Reason (R) is true.
\end{enumerate}

\item Three bells ring at intervals of $ 6, 12 and 18 minutes$. If all the three bells rang at $ 6 a.m.,$ when will they ring together again ?

\item If the system of linear equations  \\ 		
\begin{align}
		2x + 3y = 7 and \\ 
		2ax + \brak{a+b}y = 28
\end{align}
\text have infinite number of solutions, then find the values of $' a '$and$' b '$.

\item If
\begin{align}
	 217x + 131y = 913 and \\
         131x + 217y = 827,
\end{align}
 then solve the equations for the values of $x$ and $y$.

\item How many terms of the arithmetic progression $45,39,33,......$ must be taken so that their sum is $180$? Explain the double answer.
	 \item The pair of linear equations $2x = 5y + 6$ and $15y = 6x - 18$ represents two lines which are:
    \begin{enumerate}
        \item intersecting
        \item parallel
        \item coincident
        \item either intersecting or parallel
    \end{enumerate}
    \item The next term of the A.P,:$\sqrt 70$,$\sqrt 28$,$\sqrt 63$ is:
	\begin{enumerate}
        \item $\sqrt 70$
	\item $\sqrt 80$
	\item $\sqrt 97$
	\item $\sqrt 112$
\end{enumerate}
\item The roots of the equation $x^2 + 3x - 10 = 0$ are:

\begin{enumerate}
    \item$2, -5$
    \item $-2, 5$
    \item $2, 5$
    \item $-2, -5$
\end{enumerate}
\item If $\alpha, \beta$ are zeroes of the polynomial $x^2 - 1$, then the value of $\brak(\alpha + \beta)$ is:

\begin{enumerate}
    \item $2$
    \item $1$
    \item $-1$
    \item $0$
\end{enumerate}
\item If $ \alpha, \beta $ are the zeroes of the polynomial $ p\brak{x} = 4x^2 - 3x - 7 $, then $ \frac{1}{\alpha} + \frac{1}{\beta} $ is equal to:

\begin{enumerate}
    \item $\frac{7}{3}$
    \item$-\frac{7}{3}$
    \item $\frac{3}{7}$
    \item $-\frac{3}{7}$
\end{enumerate}
\end{enumerate}

